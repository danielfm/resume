\listfiles

\documentclass[11pt,a4paper]{moderncv}

% moderncv themes
\moderncvtheme[blue]{classic}

% character encoding
\usepackage[utf8]{inputenc}

% adjust page margins
\usepackage[scale=0.9]{geometry}
\recomputelengths

% for placing content in footer/header
\usepackage{fancyhdr}
\pagestyle{fancy}

% personal data
\firstname{Daniel}
\familyname{Martins}
\address{}{Rio de Janeiro, Brazil}
\title{\small Software Engineer $\star$ Open Source Enthusiast $\star$ Guitarist}
\social[github]{danielfm}
\social[skype]{danielfmt}
\social[linkedin]{danielfmartins}
\email{hey@danielfm.me}
\homepage{danielfm.me}

\begin{document}

\maketitle

\section{\textsc{Objective And Qualification Summary}}

\cvlistitem{Software engineer focused on backend systems, cluster management, operating systems, infrastructure, and related areas}
\cvlistitem{Solid experience in writing, deploying, and maintaining cloud-native applications}
\cvlistitem{Passionate about real-world high scalability problem solving}
\cvlistitem{Experience acting as a lead engineer in small teams}

\section{\textsc{Technical Skills}}

\cvlistitem{\textbf{Languages}: Go, JavaScript, Bash, Ruby, Python, Lua, Java, Clojure, Matlab, C }
\cvlistitem{\textbf{Web Frameworks}: Rails, Sinatra, Express.js, Django}
\cvlistitem{\textbf{Web Servers}: NGINX, Unicorn, Puma}
\cvlistitem{\textbf{Infrastructure as Code}: Terraform, Puppet, Capistrano, Fabric, Ansible}
\cvlistitem{\textbf{Monitoring}: Grafana, Graphite, InfluxDB, Prometheus/AlertManager}
\cvlistitem{\textbf{Container Technology}: Kubernetes, Tsuru, Docker, Rkt, Container Linux}
\cvlistitem{\textbf{IaaS/PaaS}: AWS, DigitalOcean, Heroku}
\cvlistitem{\textbf{Databases}: MongoDB, Redis, Postgres, Cassandra, MySQL}

\section{\textsc{Featured Open Source Contributions}}

\cvline{\githubsocialsymbol}{\textbf{\href{https://github.com/kubernetes/kubernetes}{kubernetes/kubernetes}}: Production-Grade Container Scheduling and Management}
\vspace*{-0.15cm}
\cvitem{}{
  {\footnotesize
    \begin{itemize}
    \item PR \href{https://github.com/kubernetes/kubernetes/pull/48707}{\#48707}: Allow nodes to create evictions for its own pods in NodeRestriction admission controller
    \end{itemize}
  }
}

\vspace*{-0.5cm}
\cvline{\githubsocialsymbol}{\textbf{\href{https://github.com/kubernetes-incubator/kube-aws}{kubernetes-incubator/kube-aws}}: Kubernetes on AWS with Container Linux}
\vspace*{-0.15cm}
\cvitem{}{
  {\footnotesize
    \begin{itemize}
     \item I'm part of the core team and active contributor, with \href{https://github.com/kubernetes-incubator/kube-aws/pulls?q=is\%3Apr+is\%3Aclosed+author\%3Adanielfm}{several merged contributions}
    \end{itemize}
  }
}

\vspace*{-0.5cm}
\cvline{\githubsocialsymbol}{\textbf{\href{https://github.com/kubernetes/ingress-nginx}{kubernetes/ingress-nginx}}: Ingress controller for NGINX}
\vspace*{-0.15cm}
\cvitem{}{
  {\footnotesize
    \begin{itemize}
    \item PR \href{https://github.com/kubernetes/ingress-nginx/pull/1238}{\#1238}: Add support for \texttt{client\_body\_timeout} and \texttt{client\_header\_timeout}
    \end{itemize}
  }
}

\vspace*{-0.5cm}
\cvline{\githubsocialsymbol}{\textbf{\href{https://github.com/kubernetes/charts}{kubernetes/charts}}: Curated applications for Kubernetes}
\vspace*{-0.15cm}
\cvitem{}{
  {\footnotesize
    \begin{itemize}
    \item PRs \href{https://github.com/kubernetes/charts/pull/533}{\#533} and \href{https://github.com/kubernetes/charts/pull/535}{\#535} for Prometheus
    \end{itemize}
  }
}

\vspace*{-0.5cm}
\section{\textsc{Featured Open Source Projects}}

\cvline{\githubsocialsymbol}{\textbf{\href{https://github.com/danielfm/prometheus-for-developers}{danielfm/prometheus-for-developers}}: Practical Prometheus workshop for developers}
\vspace*{-0.15cm}
\cvitem{}{
  {\footnotesize
    \begin{itemize}
    \item Gave this workshop for all engineering team at Descomplica for showing the basics about Prometheus
    \item Provides cookbook-style queries for common scenarios and give some tips for avoiding common pitfalls
    \end{itemize}
  }
}

\vspace*{-0.5cm}
\cvline{\githubsocialsymbol}{\textbf{\href{https://github.com/danielfm/kube-ecr-cleanup-controller}{danielfm/kube-ecr-cleanup-controller}}: ECR image cleanup controller for Kubernetes}
\vspace*{-0.15cm}
\cvitem{}{
  {\footnotesize
    \begin{itemize}
    \item Controller written in Go that deletes old Docker images from whitelisted ECR repositories
    \item It takes care not to delete images that are currently running in the cluster, and images tagged with \texttt{latest}
    \end{itemize}
  }
}

\vspace*{-0.5cm}
\cvline{\githubsocialsymbol}{\textbf{\href{https://github.com/danielfm/pybreaker}{danielfm/pybreaker}}: Circuit Breaker implementation for Python}
\vspace*{-0.15cm}
\cvitem{}{
  {\footnotesize
    \begin{itemize}
    \item Typically used in API clients in order to prevent cascading failures across multiple systems
    \item Used by \href{https://medium.com/@mauriciosl/some-tips-writing-a-tornado-application-d21ea14df51e}{Globo.com}, \href{https://youtu.be/dY-SkuENZP8?t=9m34s}{AppNeta}, \href{https://github.com/amplify-education/pybreaker}{Amplify}, \href{https://github.com/seatgeek/pybreaker}{SeatGeek}, and others
    \item Used to safeguard one of \href{http://globo.com}{Globo.com}'s the most accessed internal APIs, with tens of thousands of req/minute
    \end{itemize}
  }
}

\vspace*{-0.5cm}
\cvline{\githubsocialsymbol}{\textbf{\href{https://github.com/xmlrunner/unittest-xml-reporting}{xmlrunner/unittest-xml-reporting}}: Test results exporter for Python}
\vspace*{-0.15cm}
\cvitem{}{
  {\footnotesize
    \begin{itemize}
    \item XML file format supported by all major CI servers and build systems
    \item Used by \href{http://khanacademy.org}{Khan Academy}, \href{http://globo.com}{Globo.com}, and others
    \end{itemize}
  }
}

\cvline{\githubsocialsymbol}{\textbf{\href{https://github.com/danielfm/bencode}{danielfm/bencode}}: BitTorrent encoding implementation in Clojure}
\vspace*{-0.15cm}
\cvitem{}{
  {\footnotesize
    \begin{itemize}
    \item Parses bencoded strings directly to Clojure data structures and vice-versa
    \item Features a threadpool-based parallel piece hashing implementation that can handle tens of GiB of data
    \end{itemize}
  }
}

\section{\textsc{Professional Experience}}

\cventry{Jul. 2019 - Present}{Site Reliability Engineer}{\href{https://jusbrasil.com.br}{Jusbrasil}}{Remote from Rio de Janeiro, Brazil}{}{
  {\footnotesize 
    \begin{itemize}
    \item \textbf{Technologies got to work on}: Bash, Kubernetes, Docker, Google Cloud, NGINX, and many others
    \end{itemize}
  }
}
\vspace*{0.2cm}

\cventry{Jun. 2015 - Jul 2019}{Production Engineer / Site Reliability Engineer}{\href{https://descomplica.com.br}{Descomplica}}{Rio de Janeiro, Brazil}{}{
  {\footnotesize 
    \begin{itemize}
    \item Started to implement more formal SRE practices for monitoring Service Level Objectives (SLOs) and Error Budgets
    \item Acted as advisor for the product teams, by providing insights and architecture suggestions for making new services reliable and more well integrated with the current ecosystem
    \item Helped reduce costs by rearranging the network topology to reduce data transfer charges, as well as right-sizing the EC2 instances to better match our workload requirements, and using spot/reserved instances
    \item Changed the ingress infrastructure to leverage the AWS WAF, as well as NGINX, to mitigate common types of application-layer attacks, such as brute force and DDoS
    \item Improved security by implementing tools for enforcing basic security policies, such as disabling old access keys and removing access from users without MFA enabled
    \item Restructured user policy management by applying the principle of least privilege, role-based access control, and segregated networks for sensitive systems, such as data lakes
    \item Streamlined infrastructure management by implementing an automated pipeline for reviewing, auditing, and applying infrastructure changes
    \item Deployed a centralized logging solution for consuming the logs for all production services in a single place
    \item Created comprehensive documentation on how to mitigate and resolve all currently known failure modes
    \item Helped the product teams instrument all production services in order to give visibility into how the services were operating, and to spot optimization opportunities
    \item Set up monitoring and alerting at all layers, from the services themselves to most basic infrastructure resources, such as AWS limits and RDS database usage metrics
    \item Implemented a generic Continuous Delivery pipeline that enabled product teams to preview and safely push several changes to production every day
    \item Migrated all critical services from AWS Beanstalk to Kubernetes, significantly cutting infrastructure costs due to better resource utilization
    \item \textbf{Technologies got to work on}: Go, JavaScript, Bash, Kubernetes, Docker, AWS, NGINX, and many others
    \end{itemize}
  }
}

\vspace*{0.2cm}
\cventry{Dec. 2010 - May 2015}{Software Engineer}{\href{http://globo.com}{Globo.com}}{Rio de Janeiro, Brazil}{}{
  {\footnotesize 
    \begin{itemize}
    \item Worked on the Live Video Streaming Platform, a distributed architecture for live video ingest and delivery capable of broadcasting dozens of live streams to hundreds of thousands of users
    \item Delivered \href{http://play.com.br}{Globosat Play}, a software platform that provides cable TV subscribers access to the video content licensed in Brazil by Globosat
    \item Delivered \href{http://combate.tv}{Combate Play}, a website where Combate channel subscribers enjoy instant and unlimited access to a constantly updated collection of fights
    \item Worked on 2012's Big Brother Brasil broadcasting page, which would then become a platform for publishing live video transmission pages (used in several ocasions ever since, like the UFC fight nights, Sochi Olympics, etc)
    \item Worked on \href{http://globo.tv}{globo.tv}, a video-on-demand portal that organizes both free-to-watch and paid content produced by Globo
    \item Worked on the current version of the \href{http://g1.globo.com/fantastico/videos/}{Video Catalog}, which consists of a CMS and a pluggable application for offering video-on-demand content
    \item Delivered the first version of VideoThumbs, an API that serves dynamically resized and cropped video thumbnails
    \item \textbf{Technologies got to work on}: Ruby, Python, Lua, Redis, MongoDB, Cassandra, NGINX, and many others
    \end{itemize}
  }
}

\vspace*{0.2cm}
\cventry{Jan. 2008 - Dec. 2010}{Consultant}{Freelancer}{Ourinhos, Brazil}{}{
  {\footnotesize
    \begin{itemize}
    \item Worked as an instructor for a SCJP 1.5 course to a class of a dozen students for local development shop
    \item Delivered custom CRM system for a local real estate company that organized all brokers' activities, prospects, and kept track of sales goals
    \item Delivered a basic multilingual e-commerce website for a local real estate company
    \item Wrote a web-based user permission management tool for a client's legacy database, which were already being used by several internal applications	
    \item \textbf{Technologies got to work on}: Java, Python, Oracle, MySQL
    \end{itemize}
  }
}

\vspace*{0.2cm}
\cventry{Jun. 2004 - Sep. 2006}{Programmer}{\href{http://www.plasutil.com.br}{Plasútil}, \href{https://www.lecom.com.br/}{Lecom}}{Bauru, Brazil}{}{
  {\footnotesize
    \begin{itemize}
    \item Eliminated severe performance bottlenecks on one of their most critical product
    \item Wrote a few small Java applications for data exchange and data synchronization	
    \item Developed a new Warehouse Management System to replace the old Dataflex-based legacy system
    \item Wrote the first version of their Intranet application during my internship	
    \item \textbf{Technologies got to work on}: Java, ASP, Informix, MySQL,PHP, .Net (C\#, WinForms), Oracle
    \end{itemize}
  }
}

\vspace*{0.2cm}
\section{\textsc{Education}}

\cventry{2003 - 2006}{Bachelor, Computer Information Systems}{Faculdade Gennari \& Peartree (FGP)}{}{}{}
\cventry{2000 - 2002}{Technical School, Computer Programming}{Liceu Noroeste}{}{}{}

\section{\textsc{Courses and Certifications}}

\cventry{Mar. 2017}{LFS258: Kubernetes Fundamentals}{Linux Foundation}{}{}{}
\cventry{Dec. 2013}{CS1156X: Learning From Data}{CaltechX}{}{}{}
\cventry{Dec. 2011}{Intro to Machine Learning}{Coursera}{}{}{}
\cventry{Feb. 2007}{SCWCP: Sun Certified Web Component Developer 1.4}{Prometric}{}{}{}

\section{\textsc{Featured Articles}}

\cventry{Sep. 2017}{\href{https://danielfm.me/posts/painless-nginx-ingress.html}{Pain(less) NGINX ingress}}{Personal Website}{}{}{Post reached \#6 in \href{https://news.ycombinator.com/item?id=15238672}{HackerNews}}
\cventry{Sep. 2016}{\href{https://danielfm.me/posts/five-months-of-kubernetes.html}{Five Months of Kubernetes}}{Personal Website}{}{}{Post reached \#4 in \href{https://news.ycombinator.com/item?id=12498982}{HackerNews}}

\section{\textsc{Conferences}}

\cventry{Nov. 2016}{KubeCon}{}{Seattle/WA, USA}{}{}
\cventry{Sep. 2014}{Strange Loop}{}{St Louis/MO, USA}{}{}
\cventry{Feb. 2013}{Strata}{}{Santa Clara/CA, USA}{}{}
\cventry{Jun. 2012}{Velocity}{}{Santa Clara/CA, USA}{}{}

\cfoot{
\footnotesize{\emph{\textcolor{lightgray}{The updated version of this document can be found at https://github.com/danielfm/resume}}}
}

\end{document}
